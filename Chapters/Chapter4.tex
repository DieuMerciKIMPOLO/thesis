% Chaptre 1

\chapter{Approche suivie et solution proposée} % Main chapter title

\label{Chaptre4} % For referencing the chapter elsewhere, use \ref{Chapter1} 

Dans cette section nous ferons une présentation détaillée des solutions que nous avions proposées par rapport à nos missions.

%----------------------------------------------------------------------------------------

% Define some commands to keep the formatting separated from the content 
%\newcommand{\keyword}[1]{\textbf{#1}}
%\newcommand{\tabhead}[1]{\textbf{#1}}
%\newcommand{\code}[1]{\texttt{#1}}
%\newcommand{\file}[1]{\texttt{\bfseries#1}}
%\newcommand{\option}[1]{\texttt{\itshape#1}}
%----------------------------------------------------------------------------------------

\section{Méthodologie de travail: Scrum avec Agile}
\textbf{A compléter}
\section{Les différentes missions et solutions}
\subsection{Mission 1}
Etude de faisabilité sur le déploiement d’une application Front-End dans un environnement AWS.
L’application frontend(web app flutter) d’Assurly est hébergée sur Firebase, tandis que l’essentiel de
ses ressources se trouve dans le cloud d’Amazon. C’est dans le but d’unifier nos environnements cloud que
cette mission m’avait été confiée.
Pour cela j’avais exploré deux possibilités:
\subsubsection{Solutions}
\subparagraph{L’utilisation de S3 comme repos de déploiement}
La solution ici consiste à créer un repos S3 de le rendre public, afin de pouvoir accepter tous les trafics.
De configurer Code Pipeline pour les besoins de CI/CD avec Github, De configurer Cloudfront pour la
gestion des trafic et Route 54 pour le routage.
\subparagraph{L’utilisation d’Amplify comme comme repos de déploiement}
Cette solution consiste à utiliser le service d’hébergement qu’offre Amplify et son système de CI/CD en
l’associant à système de versioning: Github. CloudFront était utilisé pour la gestion du trafic et Route 54
pour le routage.
NB: C’est la solution 2(l’utilisation d’Amplify comme comme repos de déploiement) qui avait été
retenue. Celle-ci est simple et plus adaptée.
\subsection{Mission 2}
Le déploiement et l’intégration d’un service web (Reflex) dans un environnement AWS. La souscription
au produit d’assurance Assurly est subdivisée en trois parcours: P1, P2, et P3. Un utilisateur ne peut se
retrouver dans un seul parcours en fonction des données fournies. Si un utilisateur se retrouve dans le
parcours P3 un questionnaire plus complexe est nécessaire, c’est à cet instant qu’intervient Reflex.

\subsubsection{Solutions}
Reflex est composé de trois modules de base: CEF pour frontend, DOCS pour la gestion des rapports et
RAS qui constitue le cœur du système Reflex.
La procédure de déploiement consiste mettre à jour le repos Github qui contient les fichiers nécessaires à la
construction de notre container, à se connecter à notre EC2, à puller les fichiers du repos Github depuis un
répertoire de notre EC2 puis à construire et déployer notre container avec la commande docker-compose
\subsection{Mission 3}
Sécurisation d’une API Rest développée avec la technologie Serveless d’AWS (Lambada + python)
\subsection{Mission 4}
Le déploiement d’un système de web scraping dans un environnement AWS
