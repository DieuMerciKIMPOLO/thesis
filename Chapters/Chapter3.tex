% Chaptre 1

\chapter{Outils et technologies utilisés} % Main chapter title

\label{Chaptre3} % For referencing the chapter elsewhere, use \ref{Chapter1} 

Dans cette section je vais présenter les outils et les technologies que j’avais utilisé durant mon stage.

%----------------------------------------------------------------------------------------

% Define some commands to keep the formatting separated from the content 
%\newcommand{\keyword}[1]{\textbf{#1}}
%\newcommand{\tabhead}[1]{\textbf{#1}}
%\newcommand{\code}[1]{\texttt{#1}}
%\newcommand{\file}[1]{\texttt{\bfseries#1}}
%\newcommand{\option}[1]{\texttt{\itshape#1}}
%----------------------------------------------------------------------------------------

\section{Amazon Lambda}
AWS Lambda est un service de calcul Serverless qui vous permet d’exécuter du code sans:
\begin{list}{•}
	\item provisionner ou gérer des serveurs,
	\item créer une logique de dimensionnement de cluster prenant en charge la charge de travail,
	\item  maintenir les intégrations d’événements ou gérer les environnements d’exécution.
\end{list}
Avec Lambda, vous pouvez exécuter du code pour pratiquement n’importe quel type d’application ou service backend , sans aucune tâche administrative. Il suffit de télécharger votre code sous forme de fichier
ZIP ou d’image de conteneur, et Lambda alloue automatiquement et précisément la puissance d’exécution
de calcul et exécute votre code en fonction de la demande ou de l’événement entrant, pour n’importe quelle
échelle de trafic.
Vous pouvez configurer votre code de sorte qu’il se déclenche automatiquement depuis plus de 200 applications SaaS et services AWS, ou l’appeler directement à partir de n’importe quelle application web ou
mobile.
Vous pouvez écrire des fonctions Lambda dans votre langage préféré (Node.js, Python, Go, Java, etc.)
\subsection{Avantages}
\begin{list}{•}
	\item Aucun serveur à gérer:
	AWS Lambda exécute automatiquement votre code, sans que vous ayez à mettre en service ou à gérer des
	serveurs
	\item Dimensionnement continu:
	AWS Lambda dimensionne automatiquement votre application en exécutant le code en réponse à chaque
	déclencheur. Votre code s’exécute en parallèle et traite chaque déclencheur indépendamment. La charge de
	travail est ainsi mise à l’échelle de façon précis
	\item Optimisation des coûts grâce au comptage en millisecondes:
	Avec AWS Lambda, vous ne payez que pour le temps de calcul que vous consommez
	\item Performances constantes à n’importe quelle échelle:
	Avec AWS Lambda, vous pouvez optimiser le temps d’exécution de votre code en choisissant la bonne taille de mémoire pour
	votre fonction
\end{list}

%----------------------------------------------------------------------------------------

\section{Docker}
Docker est le système de containerisation le plus utilisé. Il permet d’embarquer une application dans un ou plusieurs containers
logiciels qui pourra s’exécuter sur n’importe quel serveur machine, qu’il soit physique ou virtuel. Docker fonctionne sous Linux
comme Windows Server. C’est une technologie qui a pour but de faciliter les déploiements d’application, et la gestion du
dimensionnement de l’infrastructure sous-jacente.
Docker possède deux type de fichier Dockerfile et docker-compose.yml:
\begin{list}{•}
	\item Dockerfile contient des instructions permettant de créer un container docker
	\item Docker-compose contient des instructions qui permettent de composer des containers
\end{list}



\section{LAmplify}

AWS Amplify est un ensemble d’outils et de services qui peuvent être utilisés ensemble ou un par un, pour
aider les développeurs web mobile et frontal à créer des applications évolutives et intégrales à technologie
AWS. Avec Amplify, vous pouvez configurer les backends d’application et connecter votre application en
quelques minutes, déployer des applications Web statiques en quelques clics et facilement gérer le contenu
des applications en dehors de la console AWS.
Amplify prend en charge les frameworks Web populaires, tels que JavaScript, React, Angular, Vue, Next.js,
et les plateformes mobiles, telles qu’Android, iOS, React Native, Ionic, Flutter.
\subsection{Fonctionnements}
\subsubsection{Developpement}
\textbf{A compléter}
\subsubsection{Hebergement}
\textbf{A compléter}


\section{Amazon Simple Storage Service (Amazon S3)}

S3 est un service de stockage d’objet offrant une évolutivité, une disponibilité des données, une sécurité et des performances de pointe. Les clients de toutes tailles et de tous secteurs peuvent ainsi utiliser
ce service afin de stocker et protéger n’importe quelle quantité de données pour un large éventail de cas
d’utilisation comme des lacs de données, des sites web, des applications mobiles, la sauvegarde et la restauration, l’archivage, des applications d’entreprise, des appareils IoT et des analyses du Big Data. Amazon
S3 fournit des fonctions de gestion faciles à utiliser pour vous permettre d’organiser vos données et de
configurer des contrôles d’accès affinés pour vos exigences métier, d’organisation et de conformité spécifiques. Amazon S3 est conçu pour offrir 99,999999999 \% de durabilité et stocker les données de millions
d’applications pour des entreprises du monde entier

\section{Aws Aurora}

Amazon Aurora est un moteur de base de données relationnelle qui associe la vitesse et la fiabilité des bases
de données commerciales haut de gamme à la simplicité et la rentabilité des bases de données open source.
Amazon Aurora MySQL offre des performances jusqu’à cinq fois supérieures à celles de MySQL sans
nécessiter de modifications de la plupart des applications MySQL. De la même manière, Amazon Aurora
PostgreSQL offre des performances jusqu’à trois fois supérieures à celles de PostgreSQL. Amazon RDS
gère vos bases de données Amazon Aurora en prenant en charge les tâches chronophages telles que la mise en service, l’application des correctifs, la sauvegarde, la récupération, la détection des pannes, ainsi que
les réparations. Vous payez un forfait mensuel pour chaque instance de base de données Amazon Aurora
utilisée. Aucun coût initial ou engagement à long terme n’est requis.

\subsection{Amazon Cognito}
Amazon Cognito permet d’ajouter facilement et rapidement l’inscription et la connexion des utilisateurs
ainsi que le contrôle d’accès aux applications Web et mobiles. Amazon Cognito s’adapte à des millions
d’utilisateurs et prend en charge la connexion avec les fournisseurs d’identité sociale tels qu’Apple, Facebook, Google et Amazon, et les fournisseurs d’identité d’entreprise via SAML 2.0 et OpenID Connect

\subsection{Fonctionnalités}
\subsubsection{Répertoire d’utilisateurs sécuritaire et se mettant à l’échelle}
Les groupes d’utilisateurs d’Amazon Cognito fournissent un répertoire d’utilisateurs sécurisé qui s’étend à
des centaines de millions d’utilisateurs. En tant que service entièrement géré, les groupes d’utilisateurs sont
faciles à configurer sans avoir à s’inquiéter de la mise en place d’une infrastructure serveur.
\subsubsection{Fédération d’identité sociale et d’entreprise}
Avec Amazon Cognito, vos utilisateurs peuvent se connecter via des fournisseurs d’identité sociale tels
que Apple, Google, Facebook et Amazon, et via des fournisseurs d’identité d’entreprise tels que SAML et
OpenID Connect.
\subsubsection{Authentification basée sur les standards}
Amazon Cognito User Pools est un fournisseur d’identités normalisé et prend en charge les normes de
gestion des identités et des accès, telles que Oauth 2.0, SAML 2.0 et OpenID Connect.
\subsubsection{Sécurité pour vos applications et vos utilisateurs}
Amazon Cognito prend en charge l’authentification multi-facteurs et le chiffrement des données au repos et
en transit. Amazon Cognito est éligible HIPAA et conforme aux normes PCI DSS, SOC, ISO/IEC 27001,
ISO/IEC 27017, ISO/IEC 27018, et ISO 9001.
\subsubsection{Contrôle d’accès pour les ressources AWS}
Amazon Cognito fournit des solutions pour contrôler l’accès aux ressources AWS à partir de votre application. Vous pouvez définir des rôles et associer des utilisateurs à des rôles différents afin que votre
application puisse accéder uniquement aux ressources autorisées pour chaque utilisateur. Autre possibilité
: vous pouvez également utiliser les attributs des fournisseurs d’identité dans les stratégies d’autorisation
AWS Identity and Access Management. Cela vous permettra de contrôler l’accès à des ressources pour les
utilisateurs qui remplissent des conditions d’attributs spécifiques
\subsubsection{Intégration facile avec votre application}
Amazon Cognito fournit des solutions pour contrôler l’accès aux ressources AWS à partir de votre application. Vous pouvez définir des rôles et associer des utilisateurs à des rôles différents afin que votre
application puisse accéder uniquement aux ressources autorisées pour chaque utilisateur. Autre possibilité
: vous pouvez également utiliser les attributs des fournisseurs d’identité dans les stratégies d’autorisation
AWS Identity and Access Management. Cela vous permettra de contrôler l’accès à des ressources pour les
utilisateurs qui remplissent des conditions d’attributs spécifiques.


\section{API Gateway}

Amazon API Gateway est un service entièrement opéré, qui permet aux développeurs de créer, publier,
gérer, surveiller et sécuriser facilement des API à n’importe quelle échelle. Les API servent de « porte
d’entrée » pour que les applications puissent accéder aux données, à la logique métier ou aux fonctionnalités de vos services backend. À l’aide d’API Gateway, vous pouvez créer des API RESTful et des API
WebSocket qui permettent de concevoir des applications de communication bidirectionnelle en temps réel.
API Gateway prend en charge les charges de travail conteneurisées et sans serveur, ainsi que les applications
web.\\
API Gateway gère toutes les tâches liées à l’acceptation et au traitement de plusieurs centaines de milliers
d’appels d’API simultanés, notamment la gestion du trafic, la prise en charge de CORS, le contrôle des
autorisations et des accès, la limitation, la surveillance et la gestion de la version de l’API. Aucuns frais
minimum ou coûts initiaux ne s’appliquent à API Gateway. Vous payez pour les appels d’API que vous
recevez et la quantité de données transférées et, avec le modèle de tarification par paliers de l’API Gateway,
vous pouvez réduire vos coûts en fonction de l’utilisation de votre API



%----------------------------------------------------------------------------------------

\section{Amazon Dynamodb}
Amazon DynamoDB est une base de données NoSql de type clé-valeur et de documents, offrant des performances de latence de l’ordre de quelques millisecondes, quelle que soit l’échelle. Il s’agit d’une base
de données multi-région, multi-active et durable entièrement gérée, avec des systèmes intégrés de sécurité,
de sauvegarde, de restauration et de mise en cache en mémoire pour les applications à l’échelle d’Internet.
DynamoDB peut traiter plus de 10 mille milliards de demandes par jour et supporte des pics de 20 millions
de demandes par seconde.\\

La plupart des entreprises du monde qui connaissent la croissance la plus rapide, comme Lyft, Airbnb et
Redfin, ainsi que Samsung, Toyota et Capital One s’appuient sur la mise à l’échelle et les performances de
DynamoDB pour prendre en charge leurs charges de travail stratégiques.
Des centaines de milliers de clients AWS ont choisi DynamoDB comme base de données de clés-valeurs et
de documents pour leurs applications mobiles, Web, de jeux, de technologie publicitaire, IoT, etc. nécessitant un accès à faible latence aux données, quelle que soit l’échelle

\section{CloudWatch}
\textbf{A compléter}
\section{Reflex}
\textbf{A compléter}
